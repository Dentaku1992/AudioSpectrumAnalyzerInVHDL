\chapter{Besluit}
	\section{Sampelen van audio}
		\par Bij het bouwen van een audio spectrum analyzer wordt al snel duidelijk dat je ergens een audiosignaal zal moeten samplen en op \'e\'en of ander manier zal moeten inlezen. Om audio binnen te lezen werd een codec ter beschikking gesteld. Deze codec werd vervolgens uitgebreid zodat het ook mogelijk werd om audio terug uit te sturen. De moeilijkheden die hierbij ondervonden werden waren vornamelijk te wijten aan de beperkte data die beschikbaar was in de datasheet. Door middel van het programma Sigma Studio was het vervolgens wel relatief eenvoudig om de juiste registerinstellingen te maken.

	\section{Dataverwerking en FFT}
		\par Om een spectrum te krijgen van een gesampled signaal moet je er mee rekenen. En een bijkomende moeilijkheid was de vereiste dat de schakeling real time moet werken. Dus met wat we toen al wisten moesten we een algoritme voor een fourier transformatie implementeren en het moest ook snel genoeg zijn om real time te zijn. Daar konden we ons beroepen op de IP library van Xilinx. Deze bevat een FFT blok, die welliswaar via AXI-bus aan te spreken is. De moeilijkheden bij de implementatie van de FFT blok beperkten zicht voornamelijk tot het juist krijgen van de timings van de verschillende signalen.

	\section{Spectrum visualiseren}
		\par Om het scherm aan te sturen waren er ook wat moeilijkheden. Op het eerste maakte het niet uit of we dan wel niet HDMI zouden gebruiken. VGA zou makkelijker zijn volgens zekere bronnen. Uiteindelijk zijn we met HDMI verder gegaan. Toch wel met dank aan een stuk voorbeeldcode gevonden op een forum zijn we dan toch aan de slag kunnen gaan. Eens we ons eerste blok op het scherm konden tekenen ging het steeds vlotter. Hoe meer we de code begrepen hoe meer we de code gingen verbouwen. Tot dat het dan weer totaal misliep, om dan een stap terug te zetten en het nog eens te proberen. 

	\section{OLED display}
		\par Het ontwerpen van de HDMI uitvoer en het OLED scherm gebeurde in twee verschillende projecten. Bij het samenvoegenvan deze projecten werd opgemerkt dat de datapin van het OLED signaal ook verbonden was met een datapin van de HDMI codec. Aanvankelijk werd er geopteerd om ofwel de data te multiplexen. De seri\"ele kloklijnen zouden dan om de beurt op 0 geschakeld worden. Uitijndelijk bleek dat de pin die verbonden was met de HDMI codec steeds op 0 stond en niet actief gebruikt werd. Deze pin maakt deel ui van een 16 bit brede data-interface waarvan enkel de 8 laatste bits in gebruik waren. De gemeenschappelijk pin bevond zich in de eerste 8, ongebruikt pinnen. Een test wees verovolgens uit dat deze data mocht overschreven worden zonder dat dit een invloed gaf op de HDMI uitvoer. Wel moest de OLED data geplaatst worden op een ODDR block omdat de HDMI pinnen via dit principe werkten. 

	\section{VHDL toolbox}
		\par Aan de hand van een kleine toolbox die ontwikkeld werd door middel van C\textsuperscript{\texttt{\#}} gemaakt werd was het op een snelle manier mogelijk om data te genereren voor de verschillende geheugens. Zo wordt onder meer tekstdata voor het OLED scherm automatisch gegenereerd en passend gemaakt voor het geheugen. Ook voor het gradi\"entgeheugen kon data gegenereerd worden op basis van een afbeelding.

	\section{Algemeen}
		\par Algemeen kunnen we stellen dat het een leerzaam project was. Gaande van niks kunnen op een fysisch FPGA development board tot een werkende audio spectrum analyzer was een vrij steile leercurve. Maar eens het eerste gelukt is volgt de rest vrij vlot. Het was vooral een kwestie van die knop om te zetten in ons hoofd. Een tweede moeilijkheid was de omgeving. Op het eerst een overdonderende indruk, een chaos aan features die naderhand uiterst praktisch bleken te zijn. We kunnen zeker stellen dat in de eerste plaats het doel, de opdracht voldaan is. Zelfs met hier een daar een uitbreiding. En in een tweede plaats heeft het zeker en vast een deur wagenwijd opengezet om zelf te gaan experimenteren met de kracht van hardware.